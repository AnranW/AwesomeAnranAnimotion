\documentclass[aspectratio=169]{beamer}
%\documentclass[compact]{beamer}
\usetheme{Hamburg}

\usepackage[T1]{fontenc}
\usepackage[utf8]{inputenc}
\usepackage{stmaryrd}
\usepackage{amsmath}
\usepackage{lmodern}

\usepackage[english]{babel}
%\usepackage[ngerman]{babel}

\usepackage{eurosym}
\usepackage{listings}
\usepackage{lstautogobble}
\usepackage{microtype}
\usepackage{textcomp}
\usepackage{units}
\DeclareSymbolFont{frenchscript}{OMS}{ztmcm}{m}{n}
%\DeclareMathSymbol{\P}{\mathord}{frenchscript}{65}

% For figures next to each other
\usepackage{graphicx}
\usepackage{graphbox}

\lstset{
	basicstyle=\ttfamily\footnotesize,
	frame=single,
	numbers=left,
	language=C,
	breaklines=true,
	breakatwhitespace=true,
	postbreak=\hbox{$\hookrightarrow$ },
	showstringspaces=false,
	autogobble=true,
	upquote=true,
	tabsize=4,
	captionpos=b,
	morekeywords={int8_t,uint8_t,int16_t,uint16_t,int32_t,uint32_t,int64_t,uint64_t,size_t,ssize_t,off_t,intptr_t,uintptr_t,mode_t}
}

\title[here is short]{Here is your Title, all Notes Special for Christian!}

\subtitle{Here is a subtitle \\ and linebreak if you like}
\author{Author 1, Author 2}
\institute{Department of Informatics\\Your Uni}
\date{\today}

% here you can have a logo, but not necessary
\titlegraphic{\includegraphics[width=0.15\textwidth]{images/onion.png}}

\begin{document}

\begin{frame}
	\titlepage
\end{frame}

\begin{frame}
	\frametitle{Agenda}
	% this hides the subsection
    % \tableofcontents[hidesubsections]
	\tableofcontents
\end{frame}

\section{Introduction}
\subsection{The name for subsection, it is hidden from every slide currently but you can change it in .sty file}

\begin{frame}[fragile]
	\frametitle{Here is the title for this frame}
	You can have text here, or have text inside of itemize:
	\begin{itemize}
	    \item You can use itemize environment to have a dot in the front. 
	    \item And you can use maximum 2 layers of itemize environment. 
	    \begin{itemize}
	        \item items here
	        \item and so much more
	        \item you can have image here or outside of the itemize environment, with or without caption
	    \end{itemize}
	\end{itemize}
	
	\begin{figure}
        \centering
        \includegraphics[width=0.3\textwidth]{images/dst.png} 
    \end{figure}
\end{frame}

\subsection{If you have multiple subsections, you will have multiple lines on top of the slide}
\begin{frame}[fragile]
	\frametitle{Here is how you have an image with captions and sources}
	\begin{figure}
	    \centering
	    \includegraphics[width=0.3\textwidth]{images/dst.png}
	    \vspace{-3mm}\caption{Caption for your image, position can be changed via $\backslash$ vspace and $\backslash$ hspace \footnote{\tiny Source: you can put your source here if you like }}
	    \label{dst}
	\end{figure}
	You can use label and reference to it: Figure \ref{dst}, but in presentations I find it unnecessary. 
\end{frame}

\section{Body}
\subsection{More on images}

\begin{frame}[fragile]
	\frametitle{If you want to have multiple images, here is a template using minipage}
	\begin{itemize}
	    \item Or you can use whatever you are used to, but don't forget to include the packages! 
	    \item But using minipage, you can also easily have half image half words, or one third, or however you like.
	\end{itemize}

    \begin{minipage}{.5\textwidth}
      \centering
      \includegraphics[width=0.7\linewidth]{images/dst.png}
    \end{minipage}%
    \begin{minipage}{.5\textwidth}
      \begin{itemize}
          \item Now there is a game that you can play!
          \begin{itemize}
              \item It is fun!
              \item You can play it with me! 
              \item Also the graphic is so cute! 
          \end{itemize}
      \end{itemize}
    \end{minipage}
\end{frame}


\section{Summary}
\subsection{}

\begin{frame}
	\frametitle{This is the summary section}
	\begin{itemize}
	    \item Intro
	    \item Body 
	    \item End
	\end{itemize}
	
	\vspace{10mm}
	\Large
	\hfill\textit{Thank you!}
\end{frame}

\begin{frame}
	\frametitle{References}
	\bibliographystyle{alpha}
	\nocite{*}
	\bibliography{literature}
\end{frame}

\end{document}